\documentclass{article}
\usepackage[utf8]{inputenc}
\usepackage[a4paper, total={6in, 8in}]{geometry}
\usepackage{amsmath, amsfonts}
\usepackage{hyperref, graphicx}

\title{CSCI 1051 Scribed Notes}
\date{}

\begin{document}

\maketitle

\section*{Motivation}

Typed class notes are a great resource for learning material and solidifying understanding.
Because I changed the topics from my last deep learning course, I don't yet have class notes but I would love your help!

In this assignment, should you choose to accept it, you will write notes for a class of your choice. You can use the following resources:
\begin{itemize}
    \item my handwritten lecture notes,
    \item the zoom recording of the lecture, and
    \item any generative AI tool of your choice.
\end{itemize}

Please sign up for a class (that hasn't yet been claimed by anyone else) on \href{https://docs.google.com/spreadsheets/d/18PWX6a_zg5qmj2239nig4vYlq4IUhKRv0tRu2yGz9_A/edit?usp=sharing}{this spreadsheet}. If no classes are available, please contact me. 

\paragraph{Note on Timing:} Even though I am normally a fan of postponing work, the benefit of an earlier topic is that the material is likely easier :)

\paragraph{Note on Grading:} I recognize that writing lecture notes is time-consuming. As such, the notes will be graded out of 9 points so that you can skip the homework on your note-taking day and still receive 100 points in the class.
On the flip side, even if you don't write notes, it is still possible to receive 95 points (an A) in the course.

\section*{Recommendations}
I recommend:
\begin{itemize}
    \item installing the Visual Studio Code application on your computer,
    \item cloning the course website from \url{https://github.com/rtealwitter/deeplearning2025/},
    \item starting from a prior note `.qmd' file in the `notes' directory,
    \item using ChatGPT or Github Copilot for help with math and formatting,
    \item writing code for figures in the `code.ipynb' file or using PowerPoint to draw code in the `images.pptx' file,
    \item once you have finished images, exporting them to the `notes/images' folder,
    \item once you're done with your notes, opening a pull request to upload the new note `.qmd' file and your images (you can do this through VSCode or a browser).
\end{itemize}

\section*{Assignment}
Please submit your pull request (and email me) by \textbf{5pm one day after your chosen class}.
So that I can easily grade, \textbf{please also submit a PDF to Gradescope.}

Your notes should cover the material in the class, explain the main concepts concisely, and showcase two images of your creation. You should use the \href{https://www.rtealwitter.com/deeplearning2025/notes/01_regression.html}{linear regression notes} and the \href{https://github.com/rtealwitter/deeplearning2025/blob/main/notes/01_regression.qmd}{corresponding source} as templates.

\begin{itemize}
    \item \textbf{Content (3 points):} Do the notes cover the class material?
    \item \textbf{Explanations (3 points):} Do the notes accessibly and concisely explain the main ideas?
    \item \textbf{New Images (3 points):} Do your images accurately and intuitively convey key concepts? 
\end{itemize}

\paragraph{Revision:} Based on your initial submission, I will give you some comments and a grade. You may resubmit the notes \textit{once} to address my comments and receive a higher grade.

\end{document}
