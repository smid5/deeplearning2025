\documentclass{article}
\usepackage[utf8]{inputenc}
\usepackage[a4paper, total={6in, 8in}]{geometry}
\usepackage{amsmath, amsfonts}
\usepackage{hyperref, graphicx}
    

\title{CSCI 1051 Project Proposal}
\author{} % Put your name here
\date{}

\begin{document}

\maketitle

\subsection*{Submission Instructions}
Please upload your proposal by \textbf{5pm Sunday January 26}.

\subsection*{Project Proposal (7 points)}

Your course project (worth 30 points total) is an opportunity
for you to explore a deep learning topic that is
interesting to you.
You can work on the project by yourself or with a partner.
The purpose of the proposal is to get you thinking 
about your project.

Peruse the topics on the homepage of the course and choose several that are interesting to you.
If you feel challenged in this course, choose topics that we will already have covered by the end of this week.
If you want more of a challenge, choose topics that we haven't yet covered.

Start brainstorming how you would go about implementing an algorithm that we discuss.
Think about:
\begin{itemize}
    \item What data sets would you run the algorithm on?
    \item How would you evaluate the algorithm?
    \item What algorithms could you compare your algorithm to?
    \item What interesting visualizations could you make describing your analysis?
    \item How could you improve the algorithm?
\end{itemize}

\subsubsection*{Deliverables}

You should submit the following in a PDF produced from
LaTeX source code:
\begin{itemize}
    \item Your name (and your partner's name if you want to work in a pair)
    \item A tentative title of your project
    \item The topic and algorithm you plan to work on
    \item answers to the questions above 
\end{itemize}

\end{document}
