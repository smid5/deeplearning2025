\documentclass{article}
\usepackage[utf8]{inputenc}
\usepackage[a4paper, total={6in, 8in}]{geometry}
\usepackage{amsmath, amsfonts, enumitem, amssymb, bm}
\usepackage{hyperref, graphicx}


\title{CSCI 1051 Homework 4}
\date{\today}

\begin{document}

\maketitle

\subsection*{Submission Instructions}

Please upload your solutions by
\textbf{5pm Friday February, 2024.}
\begin{itemize}
\item You are encouraged to discuss ideas
and work with your classmates. However, you
\textbf{must acknowledge} your collaborators
at the top of each solution on which
you collaborated with others 
and you \textbf{must write} your solutions and code independently.
\item Your solutions to theory questions must
be typeset in LaTeX or markdown.
I strongly recommend uploading the source LaTeX (found 
\href{https://www.rtealwitter.com/rads2024/psets/pset4.tex}{here})
to Overleaf for editing.
\item I recommend that you write your solutions to coding questions in a Jupyter notebook using Google Colab.
\item You should submit your solutions as a \textbf{single PDF} via the assignment on Gradescope. You can enroll in the class using the code GPXX7N.
\item Once you uploaded your solution, \textbf{mark where you answered each part of each question}.
\end{itemize}

\newpage

\section*{Problem 1}

Consider the linear regression problem with $n \geq d$.
Let $\mathbf{A} \in \mathbb{R}^{n \times d}$ be a feature matrix and $\mathbf{b} \in \mathbb{R}^n$ be a target vector.
The regression problem is to find a minimizing vector
\begin{align*}
\mathbf{x}^* = \arg \min_{\mathbf{x} \in \mathbb{R}^d}
\| \mathbf{Ax} - \mathbf{b} \|_2^2.
\end{align*}

You previously showed that the optimal solution is $\mathbf{x}^* = \left(\mathbf{A}^\top \mathbf{A} \right)^{-1} \mathbf{A}^\top \mathbf{b}$.
In this problem, you will compare computing the optimal solution exactly to computing it approximately using the fast Johnson-Lindenstrauss transform.
We will use the MNIST dataset to build $\mathbf{A}$ and $\mathbf{b}$.
The MNIST dataset consists of $28 \times 28$ pixel handrawn digits of numbers with the corresponding label.

\subsection*{Part 1 (1 point)}

Using the code I provide in \href{https://www.rtealwitter.com/rads2024/psets/regression.py}{\texttt{regression.py}}, compute the exact solution $\mathbf{x}^*$ and
the mean squared error
\begin{align*}
    \frac1{n} \| \mathbf{Ax} - \mathbf{b} \|_2^2.
\end{align*}

If your code is anything like mine, it will be slow and return a a terrible solution due to \textit{round off error}.

\subsection*{Part 2 (2 points)}
Now implement the fast JL transform as described in class.
In particular, compute $\mathbf{\Pi A} = \mathbf{S H D A}$ one column of $\mathbf{A}$ at a time.
Recall that $\mathbf{S}$ is a sampling matrix, $\mathbf{H}$ is a Hadamard, and $\mathbf{D}$ is a diagonal matrix with a random sign.

When you are done, compute the mean squared error of your solution and comment on how it compares to the ``exact'' solution.

\textbf{Hint:} Computing $\mathbf{H}$ is too expensive so write a function to compute $\mathbf{HDx}$ using recursion. You can speed up the recursion by checking if there are any non-zeros in the vector.

%\input{solutions/solution4_1}

\newpage

\section*{Problem 2}

Thank you for taking this class with me!
As I've mentioned, I love randomized algorithms for data science because the topic combines beautiful math with \emph{interesting} applications.
I know I have a lot to improve and I would love your feedback on what went well and what could have gone better!
Here are some of the aspects of the course I've thought a lot about but you can give me feedback on anything.
\begin{itemize}
    \item \textbf{Content:} What topics did you like? What would you like to have covered? What would you be okay skipping?
    \item \textbf{Difficulty:} How was the difficulty of the class? 
    \item \textbf{Daily Check in Forms}: What do you think about the daily check in forms?
    \item \textbf{Group Activities}: What did you think about the group activities? 
    \item \textbf{Content Review}: What did you think about the content review the next day?
    \item \textbf{Accessibility}: How accessible was I as a teacher? Did you feel comfortable asking me questions? Did I give enough or too many hints when asked about problems? 
    \item \textbf{Afternoon Problem Solving}: What did you think about the afternoon problem solving session? 
    \item \textbf{Self-Grade and LaTeX}: What did you think about the self-grade and writing your solutions in LaTeX?
    \item \textbf{Typed Notes and Slides}: What did you think about having the typed notes available online? How about the slides?
\end{itemize}

\section*{Part 1 (1.5 points)}
Please tell me what you liked about the class so I can do more of it in the future.

\section*{Part 2 (1.5 points)}
Please tell me what I could improve to make the experience better.

\end{document}